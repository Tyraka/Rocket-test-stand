% Created 2018-08-21 wto 12:10
% Intended LaTeX compiler: `pdflatex`
\documentclass[11pt]{article}
\usepackage[utf8]{inputenc}
\usepackage[T1]{fontenc}
\usepackage{graphicx}
\usepackage{grffile}
\usepackage{longtable}
\usepackage{wrapfig}
\usepackage{rotating}
\usepackage[normalem]{ulem}
\usepackage{amsmath}
\usepackage{textcomp}
\usepackage{amssymb}
\usepackage{capt-of}
\usepackage{hyperref}
\author{Jacob}
\date{\today}
\title{}
\hypersetup{
 pdfauthor={Jacob},
 pdftitle={},
 pdfkeywords={},
 pdfsubject={},
 pdfcreator={Emacs 26.1 (Org mode 9.1.9)},
 pdflang={English}}
\begin{document}

\tableofcontents

\section{About}
\label{sec:orgd1539f2}
This is project of DIY dynameter
\section{Amplifier - HX711}
\label{sec:org4153932}
\subsection{Amplifier schema}
\label{sec:org9eb0a06}
This is the schema for all the wireing with arduino
 \href{amplifier\_schema.png}{Schema}
\subsection{HX711 library}
\label{sec:org4babe4f}
You can download the necessary library here
\href{https://halckemy.s3.amazonaws.com/uploads/attachments/392655/HX711-master.zip}{Download}
\subsection{Arduino code}
\label{sec:org9f628ee}
Necessary code:
\\*
  - calibration
\href{calibration.int}{the code is here}
\\*
  - measurement
\href{measurement.ino}{the code is here}
\subsection{Datasheet}
\label{sec:org02fad90}
\href{https://circuits4you.com/wp-content/uploads/2016/11/hx711\_datasheet\_english.pdf}{link}
\subsection{Load cell wireing}
\label{sec:orgbceb45d}
Not important
\begin{center}
\begin{tabular}{ll}
HX711 & Load cell\\
\hline
E+ & white\\
E- & red\\
S+ & black\\
S- & green\\
\end{tabular}
\end{center}
\section{Basic load cell resistance checks}
\label{sec:org6f322a9}
\begin{center}
\begin{tabular}{ll}
Resistance check & Typical 350 Ω\\
\hline
Ex+ to Ex- & \textasciitilde{}410Ω\\
S+ to S- & 350Ω\\
Ex+ to S+ & \textasciitilde{}315Ω\\
Ex+ to S- & \textasciitilde{}315Ω\\
Ex- to S+ & \textasciitilde{}280Ω\\
Ex- to S- & \textasciitilde{}280Ω\\
\end{tabular}
\end{center}
\end{document}
